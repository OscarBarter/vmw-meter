\documentclass[11pt]{article}

\usepackage{epsfig}
\usepackage{url}

\begin{document}

\begin{center}
{\Large \bf Building the VMW Time Circuitry Meter}\\
\url{http://www.deater.net/weave/vmwprod/hardware/time_circuit/}\\
by Vincent M. Weaver\\
8 April 2014
\end{center}


\section{Introduction}

This is a work in progress.  I will update it as I complete more
of the project.


%%%%%%%%%%%%%%%%%%%%%%%%%%%%%%%%%
\pagebreak
\section{Red and Green Displays}
%%%%%%%%%%%%%%%%%%%%%%%%%%%%%%%%%

The red and green displays are separate from the yellow because
the 16-segment LEDs I could source had different pinouts, requiring
different PCB layout.

\subsection{RED/Green LED board Parts List}

\begin{tabular}{|c|l|c|c|}
\hline
Part No   &  Description    &  Quantity    &   Source \\
\hline
\hline
TIME-RGLED-MK1 & VMW Red/Green PCB Board       & 2 & VMW/OSH Park\\ %$26 each
\hline
DC56-11EWA     & Dual Red 7-segment 0.56" CC   & 5 & Kingbright\\ %??
\hline
DC56-11GWA     & Dual Green 7-segment 0.56" CC & 5 & Kingbright\\ %??
\hline
PSC05-11EWA    & 16-segment alphanum 0.5" CC   & 3 & Kingbright\\ %??
\hline
PSC05-11GWA    & 16-segment alphanum 0.5" CC   & 3 & Kingbright\\ %??
\hline
LED-1          & Red T 1 3/4 LED               & 4 & All Electronics\\ %??
\hline
LED-2          & Green T 1 3/4 LED             & 4 & All Electronics\\ %??
\hline
1427           & HT16K33 Breakout              & 2 & Adafruit\\ % $10?
\hline
SIP-30	       & 30-pin SIP header             & 10 & All Electronics\\ %5@1.20
\hline
               & 28-pin DIP socket             & 2 & ?? \\ %??
\hline
\end{tabular}

%approx $??.0 / one

\subsection{Building the Red/Green LED Board}

Note, this assumes you want to socket everything (I typically do).
You can leave out all of the sockets if you're confident in your
abilities.

\begin{enumerate}
\item Take the 28-pin DIP socket and cut it into two pieces.
      The breakout board is too wide for a regular socket.
      Solder both halves on the back of the board.
      (I don't use SIP socket here as the breakout's pins are too
       wide to fit into it).
\item Cut the SIPs into the following pieces:
\begin{itemize}
\item 19 + 9 + 2
\item 19 + 4 + 4 + 2
\item 9 + 9 + 9 + 2
\item 9 + 9 + 9 + 2
\item 9 + 9
\end{itemize}
\item Solder the two 4-length SIP to the back of the board
      for the i2c sockets
\item Solder the 9-length SIPs to the front of the board for LED
      sockets.  This is easier if the LEDs are in the sockets to
      hold them parallel.
\item Take the two 19-length SIPs and trim off the middle pin
      (measure twice cut once).  Solder these in the year spot
\item Solder the 4 2-length SIPs where the colon and AM/PM LEDs go
\item Put the various displays in the proper sockets
\item Trim the LED legs and put in the LED spots (the MK1 board
      the polarity is shown as reversed, so insert them backward)
\item Assemble the HT16K33 breakout and put in place in back.
	Solder the i2c address pins so red has address 0x74
        (addr2 soldered) and green has address 0x75 (addr0 and addr2
	soldered).
\end{enumerate}

You should be assembled!  Hook up to the i2c bus.  From above
the signals are ground, 5V, SDA, SCL.

%%%%%%%%%%%%%%%%%%%%%%%%%%%%%
\pagebreak
\section{Yellow Display}
%%%%%%%%%%%%%%%%%%%%%%%%%%%%%

The red and green displays are separate from the yellow because
the 16-segment LEDs I could source had different pinouts, requiring
different PCB layout.

The yellow display is on the bottom so unlike the red/green we
hook up the keypad socket.

\subsection{Yellow LED board Parts List}

\begin{tabular}{|c|l|c|c|}
\hline
Part No   &  Description    &  Quantity    &   Source \\
\hline
\hline
TIME-YELLOW-MK1& VMW Yellow PCB Board          & 2 & VMW/OSH Park\\ %$26 each
\hline
DC56-11YWA     & Dual Yellow 7-segment 0.56" CC& 5 & Kingbright\\ %??
\hline
PSC05-12YWA    & 16-segment alphanum 0.5" CC   & 3 & Kingbright\\ %??
\hline
LED-3          & Yellow T 1 3/4 LED            & 4 & All Electronics\\ %??
\hline
1427           & HT16K33 Breakout              & 1 & Adafruit\\ % $10?
\hline
SIP-30	       & 30-pin SIP header             & 5 & All Electronics\\ %5@1.20
\hline
               & 28-pin DIP socket             & 1 & ?? \\ %??
\hline
               & 20-pin DIP socket             & 1 & ?? \\ %??
\hline
\end{tabular}

%approx $??.0 / one

\subsection{Building the Yellow LED Board}

Note, this assumes you want to socket everything (I typically do).
You can leave out all of the sockets if you're confident in your
abilities.

\begin{enumerate}
\item Take the 28-pin DIP socket and cut it into two pieces.
      The breakout board is too wide for a regular socket.
      Solder both halves on the back of the board.
      (I don't use SIP socket here as the breakout's pins are too
       wide to fit into it).
\item Solder the 20-pin DIP socket on the bottom for the
      keypad cable.
\item Cut the SIPs into the following pieces:
\begin{itemize}
\item 19 + 9 + 2
\item 19 + 4 + 4 + 2
\item 9 + 9 + 9 + 2
\item 9 + 9 + 9 + 2
\item 9 + 9
\end{itemize}
\item Solder the two 4-length SIP to the back of the board
      for the i2c sockets
\item Solder the 9-length SIPs to the front of the board for LED
      sockets.  This is easier if the LEDs are in the sockets to
      hold them parallel.
\item Take the two 19-length SIPs and trim off the middle pin
      (measure twice cut once).  Solder these in the year spot
\item Solder the 4 2-length SIPs where the colon and AM/PM LEDs go
\item Put the various displays in the proper sockets
\item Trim the LED legs and put in the LED spots
\item Assemble the HT16K33 breakout and put in place in back
      Solder so it has i2c address 0x76 (addr2 and addr1).
\end{enumerate}

You should be assembled!  Hook up to the i2c bus.  From above
the signals are ground, 5V, SDA, SCL.

%%%%%%%%%%%%%%%%%%%%%%%%%%%%%%%%%
\pagebreak
\section{Amplifier/Speakers}
%%%%%%%%%%%%%%%%%%%%%%%%%%%%%%%%%

\subsection{Parts List}

\begin{tabular}{|c|l|c|c|}
\hline
Part No	&  Description			&  Quantity	& Source \\
\hline
\hline
	& 3.5mm audio cable		& 1		& Mouser \\
\hline
	& Dual 8 ohm speakers		& 1		& Jameco \\
\hline
987	& Stereo 3.7W Class D Audio Amplifier & 1	& Adafruit \\ %14.95
\hline
\end{tabular}

\subsection{Building}

\begin{enumerate}
\item Build the amplifier board
\item Hook 5V and ground to main power and ground
\item Hook up the audio cable ground to L- and R-,
	the left channel to L+ and right channel to R+
\item Hook up the speakers.  Left and left ground, Right and right ground
\item Plug into the pi and test.  Note you might have to be root
	for sound to play.
\end{enumerate}


%%%%%%%%%%%%%%%%%%%%%%%%%%%%%%%%%
\pagebreak
\section{Power Converter}
%%%%%%%%%%%%%%%%%%%%%%%%%%%%%%%%%

This provides 3.3V to 5V i2c conversion, allows powering from
a wall-wart, and optionally (maybe) a USB connector to provide
power to a pi so you don't need two power outlets.
Also an optional power-on LED.

\subsection{Parts List}

\begin{tabular}{|c|l|c|c|}
\hline
Part No	&  Description			&  Quantity	& Source \\
\hline
\hline
	& Power converter board		& 1		& VMW/OSHPark\\
\hline
	& USB-A connector		& 1		& Mouser \\
\hline
	& 26-pin shrouded male connector& 1		& Jameco \\
\hline
	& 5V 2A wall power supply	& 1		& ???? \\
\hline
	& 470 Ohm resistor		& 1		& ???? \\
\hline
	& LED				& 1		& ???? \\
\hline 
	& Power jack			& 1		& ???? \\
\hline
757	& 4-channel i2c bi-directional logic level converter & 1	& Adafruit \\ %$3.95
\hline
	& sockets/header pins		& ???		& ???? \\
\hline
\end{tabular}

\subsection{Building}
\begin{itemize}
\item Get the PCB
\item Put together the logic level converter
\item Solder the 26-pin socket, the power-jack, and the usb-connector into 
      place.
\item If you want a power-on indicator, solder the LED and the resistor
      into place.  *Note* on the Mark1 board the LED silkscreen is backwards.
      Optionally you can socket the LED.
\item Solder a 4-pin header into the i2c out socket
\item Solder the logic level converter to the board
\item We are always going to wall-power the device, so solder a jumper
      wire to indicate this.  *NOTE* the silkscreen is backwards on
      the Mark-1 board,
      solder from the pi-power pin to the center pin.
\end{itemize}

%%%%%%%%%%%%%%%%%%%%%%%%%%%%%%%%%%
\pagebreak
\section{Real Time Clock}
%%%%%%%%%%%%%%%%%%%%%%%%%%%%%%%%%%

This is useful to have if you ever plan to disconnect the pi from
the network.  It will keep the current time, even if you're doing
crazy things like installing time circuits into a car.

\subsection{Parts List}

\begin{tabular}{|c|l|c|c|}
\hline
Part No	&  Description			&  Quantity	& Source \\
\hline
\hline
264	& DS1307 Real Time Clock breakout & 1	& Adafruit \\ % $9.00
\hline
\end{tabular}

\subsection{Building}

\begin{enumerate}
\item Build the clock kit
\item Hook to the i2c bus
\item Configure Linux on the pi to use it (TODO)
\end{enumerate}


%%%%%%%%%%%%%%%%%%%%%%%%%%%%%%%%%
\pagebreak
\section{Keypad}
%%%%%%%%%%%%%%%%%%%%%%%%%%%%%%%%%

The keypad is not really like in the movie.  For example
it has star and pound buttons.


\subsection{Parts List}

\begin{tabular}{|c|l|c|c|}
\hline
Part No	&  Description			&  Quantity	& Source \\
\hline
\hline
	& 3x4 keypad		& 1		& All Electronics \\
\hline
	& Signal Diode		& 2		& Jameco \\
\hline
	& 39k Resistors		& 12		& ??? \\

\hline
	& Light up Switches 	& 5		& Adafruit \\
\hline
	& Case			&		& ??? \\
\hline
\end{tabular}

\subsection{Building}

\begin{enumerate}
\item Waiting until I get a PCB put together
\end{enumerate}

%%%%%%%%%%%%%%%%%%%%%%%%%%%%%%%%%%%%%%%%%%
\pagebreak
\section{Flux Capacitor}
%%%%%%%%%%%%%%%%%%%%%%%%%%%%%%%%%%%%%%%%%%

Only vaguely like the movie version and much smaller.
Also I wanted the right-angle sockets to be functional rather
than ornamental, which complicates things a bit.

\subsection{Parts List}

\begin{tabular}{|c|l|c|c|}
\hline
Part No	&  Description			&  Quantity	& Source \\
\hline
\hline
	& Waterpoof case		& 1		& Adafruit \\
\hline
FLUX-CAP-MK-1	& Circuit Board		& 1		& VMW / OSH Park \\
\hline
	& 20-pin socket			& 1		& Jameco \\
\hline
	& 30-pin SIP header		& 1		&	\\
\hline
	& Off-white LEDs		& 15		& SuperBrightLEDs \\
\hline
	& Transparent Heat-shrink Tubing &	1	& Adafruit \\
\hline
	& 1/2" standoffs		& 2		& Jameco\\
\hline
	& Screws			& 2		& Jameco\\
\hline
	& Banana Jack Sockets		& 3		& Jameco \\
\hline
	& Right-angle probe leads	& 3		& Jameco \\
\hline
\hline
\end{tabular}

\subsection{Building}

\begin{enumerate}

\item Get the circuit board.  MK1 version has an issue where the
	banana jack holes are not drilled out properly so you will
	need to drill them out to the right size.
\item Break up the SIP socket into 15 2-wide pieces.
	Solder those on for the LED sockets.
\item Solder in the 20-pin socket.
\item Place the banana jacks in place.  
	Bend the lug at a right angle so it fits better.
	Solder a bit of wire from the lug to the position on the
	circuit board.
\item Insert and cut the teast leads with 1.5 inches of wire.  Strip
	and solder the wire in place.
\item 	Cut the LED leads to 1cm.
	Place the LEDs inside the shrink wrap (haven't done this step
	yet, need more details).
\item Cut a notch on the case so the ribbon cable can get out.
	Roughly 3cm wide.  I used a flush cutter to make the notch
	for lack of better tools.
\item Optionally get an embossing label maker and make a 
	``shield your eyes'' label.
\item Permanently mount the board into the case.  The screw holes
	are some odd metric size which I didn't have screws for.
	Instead I took the standoffs (which had threaded screws coming
	off of the bottom) and super-glued them into the larger metric
	screw holes.  
	Not ideal, but worked.
\end{enumerate}

%%%%%%%%%%%%%%%%%%%%%%%%%%%%%%%%%%%%%%%%%%
\pagebreak
\section{Power Meters / Speedomoter}
%%%%%%%%%%%%%%%%%%%%%%%%%%%%%%%%%%%%%%%%%%

Todo{\dots}

\subsection{Parts List}

\begin{tabular}{|c|l|c|c|}
\hline
Part No	&  Description			&  Quantity	& Source \\
\hline
\hline
\end{tabular}

\subsection{Building}

\end{document}
