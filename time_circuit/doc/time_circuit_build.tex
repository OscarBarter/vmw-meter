\documentclass[11pt]{article}

\usepackage{epsfig}
\usepackage{url}

\begin{document}

\begin{center}
{\Large \bf Building the VMW Time Circuitry Meter}\\
\url{http://www.deater.net/weave/vmwprod/hardware/time_circuit/}\\
by Vincent M. Weaver\\
28 January 2014
\end{center}

\section{Red and Green Displays}

The red and green displays are separate from the yellow because
the 16-segment LEDs I could source had different pinouts, requiring
different PCB layout.

\subsection{RED/Green LED board Parts List}

\begin{tabular}{|c|l|c|c|}
\hline
Part No   &  Description    &  Quantity    &   Source \\
\hline
\hline
TIME-RGLED-MK1 & VMW Red/Green PCB Board       & 2 & VMW/Ohloh\\ %$26 each
\hline
DC56-11EWA     & Dual Red 7-segment 0.56" CC   & 5 & Kingsbright\\ %??
\hline
DC56-11GWA     & Dual Green 7-segment 0.56" CC & 5 & Kingsbright\\ %??
\hline
PSC05-11EWA    & 16-segment alphanum 0.5" CC   & 3 & Kingsbright\\ %??
\hline
PSC05-11GWA    & 16-segment alphanum 0.5" CC   & 3 & Kingsbright\\ %??
\hline
LED-1          & Red T 1 3/4 LED               & 4 & All Electronics\\ %??
\hline
LED-2          & Green T 1 3/4 LED             & 4 & All Electronics\\ %??
\hline
1427           & HT16K33 Breakout              & 2 & Adafruit\\ % $10?
\hline
SIP-30	       & 30-pin SIP header             & 10 & All Electronics\\ %5@1.20
\hline
               & 28-pin DIP socket             & 2 & ?? \\ %??
\hline
\end{tabular}

%approx $??.0 / one

\subsection{Building the Red/Green LED Board}

Note, this assumes you want to socket everything (I typically do).
You can leave out all of the sockets if you're confident in your
abilities.

\begin{enumerate}
\item Take the 28-pin DIP socket and cut it into two pieces.
      The breakout board is too wide for a regular socket.
      Solder both halves on the back of the board.
      (I don't use SIP socket here as the breakout's pins are too
       wide to fit into it).
\item Cut the SIPs into the following pieces:
\begin{itemize}
\item 19 + 9 + 2
\item 19 + 4 + 4 + 2
\item 9 + 9 + 9 + 2
\item 9 + 9 + 9 + 2
\item 9 + 9
\end{itemize}
\item Solder the two 4-length SIP to the back of the board
      for the i2c sockets
\item Solder the 9-length SIPs to the front of the board for LED
      sockets.  This is easier if the LEDs are in the sockets to
      hold them parallel.
\item Take the two 19-length SIPs and trim off the middle pin
      (measure twice cut once).  Solder these in the year spot
\item Solder the 4 2-length SIPs where the colon and AM/PM LEDs go
\item Put the various displays in the proper sockets
\item Trim the LED legs and put in the LED spots (the MK1 board
      the polarity is shown as reversed, so insert them backward)
\item Assemble the HT16K33 breakout and put in place in back.
	Red solder so address 0x74, green 0x75
\end{enumerate}

You should be assembled!  Hook up to the i2c bus.  From above
the signals are ground, 5V, SDA, SCL.



\section{Yellow Display}

The red and green displays are separate from the yellow because
the 16-segment LEDs I could source had different pinouts, requiring
different PCB layout.

The yellow display is on the bottom so unlike the red/green we
hook up the keypad socket.

\subsection{Yellow LED board Parts List}

\begin{tabular}{|c|l|c|c|}
\hline
Part No   &  Description    &  Quantity    &   Source \\
\hline
\hline
TIME-YELLOW-MK1& VMW Yellow PCB Board          & 2 & VMW/Ohloh\\ %$26 each
\hline
DC56-11YWA     & Dual Yellow 7-segment 0.56" CC& 5 & Kingsbright\\ %??
\hline
PSC05-12YWA    & 16-segment alphanum 0.5" CC   & 3 & Kingsbright\\ %??
\hline
LED-3          & Yellow T 1 3/4 LED            & 4 & All Electronics\\ %??
\hline
1427           & HT16K33 Breakout              & 2 & Adafruit\\ % $10?
\hline
SIP-30	       & 30-pin SIP header             & 5 & All Electronics\\ %5@1.20
\hline
               & 28-pin DIP socket             & 1 & ?? \\ %??
\hline
               & 20-pin DIP socket             & 1 & ?? \\ %??
\hline
\end{tabular}

%approx $??.0 / one

\subsection{Building the Yellow LED Board}

Note, this assumes you want to socket everything (I typically do).
You can leave out all of the sockets if you're confident in your
abilities.

\begin{enumerate}
\item Take the 28-pin DIP socket and cut it into two pieces.
      The breakout board is too wide for a regular socket.
      Solder both halves on the back of the board.
      (I don't use SIP socket here as the breakout's pins are too
       wide to fit into it).
\item Solder the 20-pin DIP socket on the bottom for the
      keypad cable.
\item Cut the SIPs into the following pieces:
\begin{itemize}
\item 19 + 9 + 2
\item 19 + 4 + 4 + 2
\item 9 + 9 + 9 + 2
\item 9 + 9 + 9 + 2
\item 9 + 9
\end{itemize}
\item Solder the two 4-length SIP to the back of the board
      for the i2c sockets
\item Solder the 9-length SIPs to the front of the board for LED
      sockets.  This is easier if the LEDs are in the sockets to
      hold them parallel.
\item Take the two 19-length SIPs and trim off the middle pin
      (measure twice cut once).  Solder these in the year spot
\item Solder the 4 2-length SIPs where the colon and AM/PM LEDs go
\item Put the various displays in the proper sockets
\item Trim the LED legs and put in the LED spots
\item Assemble the HT16K33 breakout and put in place in back
      Solder so address 0x76
\end{enumerate}

You should be assembled!  Hook up to the i2c bus.  From above
the signals are ground, 5V, SDA, SCL.


\end{document}
