\documentclass[11pt]{article}
\begin{document}

Building the Super-VMW CPU Meter

\section{Parts List}

\begin{tabular}{|c|l|c|c|}
\hline
Part No   &  Description    &  Quantity    &   Source \\
\hline
\hline

          &  LED Board                 & 1  & VMW SVMW-Meter-LED-Board\\ %$15
LTP 3786E &  14-Seg 2x Alphanum Com Anode & 3 &  DigiKey 160-1011-ND \\ % 3@3.78
% Jameco also has a selection of colors.  Blue?
LTA 1000G &  10-segment bargraph       & 2  &  Jameco 697469 \\ %2@1.19
          &  Mini Red LED              & 12 &  Jameco 333851 \\ %12@.12
	  &  1200mcd R,O,Y,G,B,P LEDs  & 6  &  Super Bright LED \\ % 6@.60
	  &  20-pin .3 wide DIP socket & 4  &  Jameco 38608 \\ %4@.17
	  &  30-pin SIP header         & 5  &  All-Electronics SIP-30 \\ %5@1.20
\hline
\hline
          & Logic Board                & 1  & VMW SVMW-Meter-Logic-Board \\ %$20 
	  & DC Male 2.1mm Solder Jack  & 1  &  Jameco 101178 \\ %0.39
	  & 25-pin right-angle DSUB male & 1  &  Jameco 15150 \\ %0.75
	  & 24-pin .6 wide DIP socket  & 4  &  Jameco 112264 \\ %4@.23
	  & 20-pin .3 wide DIP socket  & 4  &  Jameco 38608 \\ %4@.17
	  & 16-pin DIP socket          & 1  &  Jameco 112222 \\ %1@.19
	  & 4.7K Resistor              & 7  &  Jameco 691024 \\ %7@.02
	  & 27K Resistor               & 1  &  Jameco 691201 \\ %.02
	  & 16K Resistor               & 1  &  Jameco 691155 \\ %.02
	  & 3.3K Resistor              & 1  &  Jameco 690988 \\ %.02
	  & 2.7nF Capacitor            & 4  &  Jameco 544905 \\ %10@.15
	  & 0.1$\mu$F Capacitor        & 4  &  Jameco 15229 \\ %10@0.06
PN2222    & NPN Transistor             & 8  &  Jameco 28628 \\ %10@0.07
SAA1064   & LED Driver w I$^{2}$C      & 4  &  DigiKey 568-1107-5-ND \\ % 4@2.79
74HC05    & Inverter w/ Open Drain     & 1  &  DigiKey 296-1568-5-ND \\ % 1@.57


\hline
\hline
          & SERPAC A-31 Case           & 1   & Digikey SRA31G-ND \\ % 1@9.54
	  & 20-conductor Flat 28AWG wire & 16in & Jameco 643874 \\ % $3.95
          & IDC 20-pin DIP cable mount & 8   & Jameco 104097 \\ % 8@.65
	  & 5V 1A power adapter        & 1   & Jameco S08AA05010001 \\ %12.95
	  & Parallel Cable             & 1   & \\
HT-214D   & IDC DIP Crimping Tool      & 1   & Newark 69C1030 \\ % 1@15.62
\hline

\end{tabular}

%approx $110.0 / one

\section{Building the LED Board}

\begin{enumerate}
\item Take two 30-pin headers.  Cut off 4 10-pin lengths for
      the LED bargraph sockets.  (These need to be made out
      of the headers, so that it is possible to solder the
      sockets from the other side).  Solder these to the board.
      Put a 20-pin socket into the headers to keep them vertical.
\item Solder the 4 20-pin sockets on the back of the board.
      The one that comes up between the bargraph sockets is a bit tricky.
\item Take a 30-pin header and snip pins 10 and 20 off, plus trim off
      the 30th pin completely.  Repeat.  Solder these on for the
      alphanumeric sockets.  Place a wide 24-pin socket across while
      soldering to keep vertical.
\item Cut 22 2-pin lengths of the 30-pin headers for LED sockets.
      Solder them in place.  Scotch tape can hold them in place when
      soldering.
\item Place all of the LED displays in place.  
      {\bf WORKAROUND} If you are using the
      Mark2 board the 12 red LEDs are indicated backwards, you'll
      have to install them opposite the way indicated on the PCB.
\item {\bf WORKAROUND} If you have the Mark2 board then the
      cable connectors are too close together to fit in the socket.
      As a workaround build a stack of cut up 30-pin headers 8 wide
      and place on the middle connector.
      Stack 3 high.  Then solder to the far right and left pins wires
      up to a 10-wide stack at top. This will allow the side connectors
      to go on, as well as the middle one to be lofted.
\end{enumerate}

\section{Building the Logic Board}
\begin{enumerate}
\item Solder the sockets in place.  First the 4 24-pin wide ones.
      Then the 4 20-pin sockets.  Then the 1 16-pin socket.
\item You can optionally use parts of header to make i2c and voltage
      interface sockets.
\item Bend the transistors down (so they lie flush) so that the
      cable headers can fit on the sockets w/o interruption.
\item Solder the resistors in place.  
      {\bf WORKAROUND} The 27K resistor is mislabeled
      21K on the board.
\item Solder the capacitors in place.
\item Solder the power socket.
\item Solder on the parallel connector.

\item {\bf WORKAROUND} Loft the middle SAA1064 chip.  Use two 10-wide
      headers to loft it.  Then solder wires to the holes on either
      end and loft.  You might insert heat-shrink tubing around wires
      to keep shorts from happening.
\item Insert the 74HC05 chip, the four SAA1064 chips.

\end{enumerate}

\section{Assembling the Whole Thing}
\begin{enumerate}
\item {\bf WORKAROUND} If the left side of the board won't fit in the
      case, you'll need to file a semi-circular hole so that the board
      will fit.
\item Screw the logic board to base using two of the screws that
      came with the case.
\item Crimp the 4 cables with 20-pin connectors on either end.
      A crimper really helps here, and you really need one that
      had a DIP crimping substrate (hard to find).
      From the front, the left cable is 5in, next 2.5in (longer
      if you don't need the workaround), 3in, then 5.5in.      
\item Connect cables.  {\bf WORKAROUND}  If using mark1 logic board,
      need to trim the side off of the right-side connector
      so it can fit and not conflict with bottom right connector.
\item Plug in 5V adapter and parallel port.  Test using one
     of the sample programs.
\end{enumerate}

\section{Testing}
\begin{enumerate}
\item Before inserting ICs, apply 5V power and check that power is
      at proper pins.
\item Try inserting SAA1064s one at a time and then running sample
      programs to make sure each works/no problems before inserting all.
\end{enumerate}


\section{Optional USB Board}
\begin{enumerate}
\item Investigate putting together usb-tiny board for use on machines
      without parallel port.
\end{enumerate}

\end{document}
